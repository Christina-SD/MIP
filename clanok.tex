% Metódy inžinierskej práce

\documentclass[10pt,twoside,slovak,a4paper]{article}

\usepackage[slovak]{babel}
%\usepackage[T1]{fontenc}
\usepackage[IL2]{fontenc} % lepšia sadzba písmena Ľ než v T1
\usepackage[utf8]{inputenc}
\usepackage{graphicx}
\usepackage{url} % príkaz \url na formátovanie URL
\usepackage{hyperref} % odkazy v texte budú aktívne (pri niektorých triedach dokumentov spôsobuje posun textu)

\usepackage{cite}
%\usepackage{times}

\pagestyle{headings}

\title{The development of Dota 2, features of the game\thanks{Semestrálny projekt v predmete Metódy inžinierskej práce, ak. rok 2022/23, vedenie: //// }} % meno a priezvisko vyučujúceho na cvičeniach

\author{Khrystyna Bindiuk\\[2pt]
	{\small Slovenská technická univerzita v Bratislave}\\
	{\small Fakulta informatiky a informačných technológií}\\
	{\small \texttt{xbindiuk@stuba.sk}}
	}

\date{\small 4. november 2022} % upravte



\begin{document}

\maketitle

\begin{abstract}

Nowadays computer games are widely used by various age groups ranging from children to adults. A majority of users prefer the genre of action games\cite{Research:Top}. One of the most popular action games is Dota 2 in accordance with an ongoing analysis of Steam's concurrent players \cite{Steam:Dota}. As a consequence, the main purpose of the article is to describe the phases of Dota 2 development, the main features of the game, and the factors that caused its popularity. The paper consists of several parts such as the development phases of Dota 2, the main features of Dota 2, and the reasons why Dota 2 is still popular
\end{abstract}



\section{Introduction}

Computer games become indispensable parts of daily lives in recent decades. It is caused by the global digitalization of society and the irreversible penetration of state-of-the-art technologies into the market. As a result, there is a growing demand for products that may capture the attention of possible users. To achieve the above-mentioned goal, game developers introduce fresh and unusual scenarios and heroes in their programs. One of those games is Dota 2. 

Dota 2 is a multiplayer online battle arena game that was created by the American video game developer company Valve Corporation. The product is known for its isometric perspective, 3D graphics and wide range of modes. Moreover, players have the possibility to choose various heroes because there are 120 characters that users can take from before the beginning of a battle.

The article covers the phases of Dota 2 development in the section~\ref{The development phases of Dota 2}. Furthermore, the main characteristics of the game and factors that have an impact on Dota 2 popularity are described in~\ref{dolezita} and~\ref{dolezitejsia}. The concluding remarks are provided in the part~\ref{zaver}.



\section{The development phases of Dota 2} \label{The development phases of Dota 2}
It is claimed that the development path of Dota 2 consists of several phases.

First and foremost, the concept of the game originated from the Aeon of Strife fan mod for StarCraft: Brood War in 1998\cite{Stubbs:Rise}. The interesting point in the Aeon of Strife is that the three-lane system of creeps was the impetus for the development of Dota 2. 

Moving forward, modder Kyle “Eul” Sommer created a mod for Warcraft 3 called Defense of the Ancients or DotA in the early 2000s. The main difference with the modern mod of Dota 2 is an engine that misses bells and whistles. Other features are similar to Dota 2. To some extent, there are five players per team that compete against five others on a map with three lanes and try to destroy the base of an opposite team. At the same time, each unique character has distinguishable abilities and properties. 

Next, new versions of the mod appeared because the code of the map was opened for everyone. One of the most popular among them was Dota Allstars\cite{Pra:What} which was developed by Steve “Guinsoo”. Therefore, Dota Allstars became the prototype of Dota 2.

Finally, Valve Corporation decided to work on a project with game developer IceFrog in 2009. The game was released in beta form in 2011 and gained popularity among players from around the globe. Later, it was officially published on Steam in mid-2013. In addition, in 2015 there was a “Reborn” update that completely changed the User Interface of the game. The next update called the 7.00 update was in 2016, and it is regarded as the greatest transformation of core Dota 2 gameplay.




\section{The main features of Dota 2} \label{The main features of Dota 2}


Dota 2 is known for its huge range of features. Thus, the section is divided into an overview of Dota 2~\ref{overview}, its main modes(časť~\ref{ina:nejake}), positions(časť~\ref{ina:nejake}) and heroes(časť~\ref{ina:nejake}). 


\subsection{An overview of Dota 2} \label{overview}


Each team has five players. If a team invades the opponent’s base and destroys the Ancestor, it is considered a victory. Each player has a specific position and one person is designated as captain. The captains of each team have the possibility to choose five heroes, who will be in their team. During the draft, a captain can ban some heroes from usage by opponents. Then, team members select heroes from remained options. When a person chooses a character, he needs to pay attention to the individual strengths and weaknesses of it as well as the strengths and weaknesses of that hero in combination with characters already selected by other team members. [1]

Furthermore, an economic system is available in the game. To explicate, a user can earn gold when he kills a creep of another team. Therefore, if a player wants to become stronger, he needs to earn more gold and buy special items that boost a hero.
In Dota 2, there are also “neutral creeps” which are outside of the lanes and forest monsters. An interesting aspect of the game is a creature called “Roshan” who is capable of changing fate. After a player killed Roshan, he gets the item that can revive a person after death.



\subsection{Ešte nejaké vysvetlenie} \label{ina:este}

\paragraph{Veľmi dôležitá poznámka.}
Niekedy je potrebné nadpisom označiť odsek. Text pokračuje hneď za nadpisom.



\section{Dôležitá časť} \label{dolezita}




\section{Ešte dôležitejšia časť} \label{dolezitejsia}




\section{Záver} \label{zaver} % prípadne iný variant názvu



%\acknowledgement{Ak niekomu chcete poďakovať\ldots}


% týmto sa generuje zoznam literatúry z obsahu súboru literatura.bib podľa toho, na čo sa v článku odkazujete
\bibliography{literatura}
\bibliographystyle{plain} % prípadne alpha, abbrv alebo hociktorý iný
\end{document}
